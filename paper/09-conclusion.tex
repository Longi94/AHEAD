% !TeX encoding = UTF-8
% !TeX root = sigmod2018.tex
% !TeX spellcheck = en_US
\section{Conclusion and Outlook}
\label{sec:Conclusion}

Future hardware becomes less reliable in total and scaling up todays hardware-based protection introduces too much overhead~~\cite{DBLP:journals/micro/Borkar05,DBLP:conf/dac/HenkelBDGNSTW13,DBLP:books/daglib/0037372,DBLP:journals/it/ShafiqueABCCDEH15}. Therefore, a shift towards mitigating these reliability issues at higher layers, rather than only dealing with these issues in hardware was initiated~\cite{DBLP:conf/dac/HenkelBDGNSTW13,DBLP:books/daglib/0037372,DBLP:journals/it/ShafiqueABCCDEH15}. However, traditional general-purpose software-based protection techniques mainly rely on dual modular redundancy (DMR) to detect errors~\cite{goloubeva2006software,oh2002error,DBLP:conf/cgo/ReisCVRA05,DBLP:books/daglib/0037372}. For database systems, DMR introduces high overhead, because all data has to be duplicated and every query is executed redundantly including a result comparison as the final step. To overcome these drawbacks, we have presented our novel \emph{adaptable and on-the-fly \textbf{error detection} approach} called \emph{AHEAD}. With our approach, we achieve the following properties: (1) \emph{AHEAD} detects (i) errors (multi\=bit flips) that modify data stored in main memory or transmit over an interconnect and (ii) errors induced during computations, (2) \emph{AHEAD} provides configurable error detection capabilities to be able to adapt to different error models at run-time, and (3) \emph{AHEAD} drastically reduces the overhead compared to DMR and errors are continuously detected at query processing. Thus, \emph{AHEAD} is the first comprehensive database-specific approach to tackle the challenge of resilient query processing on unreliable hardware. As next, the following steps have to be done: 

\textbf{Optimization and Extension:} In Section~\ref{sec:SSBEval} we show that our current storage overhead is suboptimal. Bit-level data compression as in~\cite{Willhalm:2009:SUF:1687627.1687671,willhalm2013vectorizing} could be one solution. While data hardening and lightweight compression~\cite{DBLP:conf/sigmod/AbadiMF06,DBLP:conf/edbt/DammeHHL17} are orthogonal to each other, their interplay is very important to keep the overall memory footprint of data as low as possible. With hardening, compression gains even more significance, since it can reduce the newly introduced storage overhead. However, combining both may be challenging and has to be investigated in detail. Furthermore, \emph{AHEAD} cannot detect errors in logic operations, whereas these are frequently used in database systems, e.g., in novel column storage layouts like BitWeaving~\cite{Li:2013:BFS:2463676.2465322} and ByteSlice~\cite{Feng:2015:BPE:2723372.2747642}. This domain must be considered separately, for this, \emph{AHEAD} can serve as the basis. Further extensions of \emph{AHEAD} could be (1) the use of code word accumulators to do detection every \(n\)th code word, trading accuracy against performance, or (2) hardening of database meta data and block or string data. %Moreover, our \emph{AHEAD} approach should be investigated in multi-threaded environments for more scalability considerations. 


\textbf{Error Correction:} So far, we were concerned with the continuous detection of bit flips during query processing. Next, continuous error correction should be considered so that detected bit flips are corrected during query processing. With \emph{AHEAD}, we are able detect bit flips on value granularity and can find out where the error occurred. Based on that, specific correction techniques can be developed and integrated in the query processing. For example, if we detect a faulty code word in the inputs of an operator, we can retransmit it, possibly several times, to correct errors induced during transmission. If we get a valid code word, processing can continue with this correct code word. If we get an invalid code word, we can assume that bits are flipped in main memory and then we require an appropriate technique for error correction. For that, correcting errors in the memory requires data redundancy in any case. %Here, classical RAID-techniques or techniques from the network coding domain may be interesting to consult. 

\textbf{Cross-Layer Approach:} AHEAD is primarily a software approach. Another interesting research direction would examine the interplay of hardware and software protection mechanisms. In particular it should be scrutinized what should be done in hardware and what should be done in software. From our point of view, \emph{AHEAD} could serve as foundation for such a novel research direction~\cite{DBLP:books/daglib/0037372}.

